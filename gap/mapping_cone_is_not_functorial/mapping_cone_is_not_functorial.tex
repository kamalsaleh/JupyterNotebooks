\documentclass[11pt]{article}

    \usepackage[breakable]{tcolorbox}
    \usepackage{parskip} % Stop auto-indenting (to mimic markdown behaviour)
    
    \usepackage{iftex}
    \ifPDFTeX
    	\usepackage[T1]{fontenc}
    	\usepackage{mathpazo}
    \else
    	\usepackage{fontspec}
    \fi

    % Basic figure setup, for now with no caption control since it's done
    % automatically by Pandoc (which extracts ![](path) syntax from Markdown).
    \usepackage{graphicx}
    % Maintain compatibility with old templates. Remove in nbconvert 6.0
    \let\Oldincludegraphics\includegraphics
    % Ensure that by default, figures have no caption (until we provide a
    % proper Figure object with a Caption API and a way to capture that
    % in the conversion process - todo).
    \usepackage{caption}
    \DeclareCaptionFormat{nocaption}{}
    \captionsetup{format=nocaption,aboveskip=0pt,belowskip=0pt}

    \usepackage[Export]{adjustbox} % Used to constrain images to a maximum size
    \adjustboxset{max size={0.9\linewidth}{0.9\paperheight}}
    \usepackage{float}
    \floatplacement{figure}{H} % forces figures to be placed at the correct location
    \usepackage{xcolor} % Allow colors to be defined
    \usepackage{enumerate} % Needed for markdown enumerations to work
    \usepackage{geometry} % Used to adjust the document margins
    \usepackage{amsmath} % Equations
    \usepackage{amssymb} % Equations
    \usepackage{textcomp} % defines textquotesingle
    % Hack from http://tex.stackexchange.com/a/47451/13684:
    \AtBeginDocument{%
        \def\PYZsq{\textquotesingle}% Upright quotes in Pygmentized code
    }
    \usepackage{upquote} % Upright quotes for verbatim code
    \usepackage{eurosym} % defines \euro
    \usepackage[mathletters]{ucs} % Extended unicode (utf-8) support
    \usepackage{fancyvrb} % verbatim replacement that allows latex
    \usepackage{grffile} % extends the file name processing of package graphics 
                         % to support a larger range
    \makeatletter % fix for grffile with XeLaTeX
    \def\Gread@@xetex#1{%
      \IfFileExists{"\Gin@base".bb}%
      {\Gread@eps{\Gin@base.bb}}%
      {\Gread@@xetex@aux#1}%
    }
    \makeatother

    % The hyperref package gives us a pdf with properly built
    % internal navigation ('pdf bookmarks' for the table of contents,
    % internal cross-reference links, web links for URLs, etc.)
    \usepackage{hyperref}
    % The default LaTeX title has an obnoxious amount of whitespace. By default,
    % titling removes some of it. It also provides customization options.
    \usepackage{titling}
    \usepackage{longtable} % longtable support required by pandoc >1.10
    \usepackage{booktabs}  % table support for pandoc > 1.12.2
    \usepackage[inline]{enumitem} % IRkernel/repr support (it uses the enumerate* environment)
    \usepackage[normalem]{ulem} % ulem is needed to support strikethroughs (\sout)
                                % normalem makes italics be italics, not underlines
    \usepackage{mathrsfs}
    

    
    % Colors for the hyperref package
    \definecolor{urlcolor}{rgb}{0,.145,.698}
    \definecolor{linkcolor}{rgb}{.71,0.21,0.01}
    \definecolor{citecolor}{rgb}{.12,.54,.11}

    % ANSI colors
    \definecolor{ansi-black}{HTML}{3E424D}
    \definecolor{ansi-black-intense}{HTML}{282C36}
    \definecolor{ansi-red}{HTML}{E75C58}
    \definecolor{ansi-red-intense}{HTML}{B22B31}
    \definecolor{ansi-green}{HTML}{00A250}
    \definecolor{ansi-green-intense}{HTML}{007427}
    \definecolor{ansi-yellow}{HTML}{DDB62B}
    \definecolor{ansi-yellow-intense}{HTML}{B27D12}
    \definecolor{ansi-blue}{HTML}{208FFB}
    \definecolor{ansi-blue-intense}{HTML}{0065CA}
    \definecolor{ansi-magenta}{HTML}{D160C4}
    \definecolor{ansi-magenta-intense}{HTML}{A03196}
    \definecolor{ansi-cyan}{HTML}{60C6C8}
    \definecolor{ansi-cyan-intense}{HTML}{258F8F}
    \definecolor{ansi-white}{HTML}{C5C1B4}
    \definecolor{ansi-white-intense}{HTML}{A1A6B2}
    \definecolor{ansi-default-inverse-fg}{HTML}{FFFFFF}
    \definecolor{ansi-default-inverse-bg}{HTML}{000000}

    % commands and environments needed by pandoc snippets
    % extracted from the output of `pandoc -s`
    \providecommand{\tightlist}{%
      \setlength{\itemsep}{0pt}\setlength{\parskip}{0pt}}
    \DefineVerbatimEnvironment{Highlighting}{Verbatim}{commandchars=\\\{\}}
    % Add ',fontsize=\small' for more characters per line
    \newenvironment{Shaded}{}{}
    \newcommand{\KeywordTok}[1]{\textcolor[rgb]{0.00,0.44,0.13}{\textbf{{#1}}}}
    \newcommand{\DataTypeTok}[1]{\textcolor[rgb]{0.56,0.13,0.00}{{#1}}}
    \newcommand{\DecValTok}[1]{\textcolor[rgb]{0.25,0.63,0.44}{{#1}}}
    \newcommand{\BaseNTok}[1]{\textcolor[rgb]{0.25,0.63,0.44}{{#1}}}
    \newcommand{\FloatTok}[1]{\textcolor[rgb]{0.25,0.63,0.44}{{#1}}}
    \newcommand{\CharTok}[1]{\textcolor[rgb]{0.25,0.44,0.63}{{#1}}}
    \newcommand{\StringTok}[1]{\textcolor[rgb]{0.25,0.44,0.63}{{#1}}}
    \newcommand{\CommentTok}[1]{\textcolor[rgb]{0.38,0.63,0.69}{\textit{{#1}}}}
    \newcommand{\OtherTok}[1]{\textcolor[rgb]{0.00,0.44,0.13}{{#1}}}
    \newcommand{\AlertTok}[1]{\textcolor[rgb]{1.00,0.00,0.00}{\textbf{{#1}}}}
    \newcommand{\FunctionTok}[1]{\textcolor[rgb]{0.02,0.16,0.49}{{#1}}}
    \newcommand{\RegionMarkerTok}[1]{{#1}}
    \newcommand{\ErrorTok}[1]{\textcolor[rgb]{1.00,0.00,0.00}{\textbf{{#1}}}}
    \newcommand{\NormalTok}[1]{{#1}}
    
    % Additional commands for more recent versions of Pandoc
    \newcommand{\ConstantTok}[1]{\textcolor[rgb]{0.53,0.00,0.00}{{#1}}}
    \newcommand{\SpecialCharTok}[1]{\textcolor[rgb]{0.25,0.44,0.63}{{#1}}}
    \newcommand{\VerbatimStringTok}[1]{\textcolor[rgb]{0.25,0.44,0.63}{{#1}}}
    \newcommand{\SpecialStringTok}[1]{\textcolor[rgb]{0.73,0.40,0.53}{{#1}}}
    \newcommand{\ImportTok}[1]{{#1}}
    \newcommand{\DocumentationTok}[1]{\textcolor[rgb]{0.73,0.13,0.13}{\textit{{#1}}}}
    \newcommand{\AnnotationTok}[1]{\textcolor[rgb]{0.38,0.63,0.69}{\textbf{\textit{{#1}}}}}
    \newcommand{\CommentVarTok}[1]{\textcolor[rgb]{0.38,0.63,0.69}{\textbf{\textit{{#1}}}}}
    \newcommand{\VariableTok}[1]{\textcolor[rgb]{0.10,0.09,0.49}{{#1}}}
    \newcommand{\ControlFlowTok}[1]{\textcolor[rgb]{0.00,0.44,0.13}{\textbf{{#1}}}}
    \newcommand{\OperatorTok}[1]{\textcolor[rgb]{0.40,0.40,0.40}{{#1}}}
    \newcommand{\BuiltInTok}[1]{{#1}}
    \newcommand{\ExtensionTok}[1]{{#1}}
    \newcommand{\PreprocessorTok}[1]{\textcolor[rgb]{0.74,0.48,0.00}{{#1}}}
    \newcommand{\AttributeTok}[1]{\textcolor[rgb]{0.49,0.56,0.16}{{#1}}}
    \newcommand{\InformationTok}[1]{\textcolor[rgb]{0.38,0.63,0.69}{\textbf{\textit{{#1}}}}}
    \newcommand{\WarningTok}[1]{\textcolor[rgb]{0.38,0.63,0.69}{\textbf{\textit{{#1}}}}}
    
    
    % Define a nice break command that doesn't care if a line doesn't already
    % exist.
    \def\br{\hspace*{\fill} \\* }
    % Math Jax compatibility definitions
    \def\gt{>}
    \def\lt{<}
    \let\Oldtex\TeX
    \let\Oldlatex\LaTeX
    \renewcommand{\TeX}{\textrm{\Oldtex}}
    \renewcommand{\LaTeX}{\textrm{\Oldlatex}}
    % Document parameters
    % Document title
    \title{mapping\_cone\_is\_not\_functorial}
    
    
    
    
    
% Pygments definitions
\makeatletter
\def\PY@reset{\let\PY@it=\relax \let\PY@bf=\relax%
    \let\PY@ul=\relax \let\PY@tc=\relax%
    \let\PY@bc=\relax \let\PY@ff=\relax}
\def\PY@tok#1{\csname PY@tok@#1\endcsname}
\def\PY@toks#1+{\ifx\relax#1\empty\else%
    \PY@tok{#1}\expandafter\PY@toks\fi}
\def\PY@do#1{\PY@bc{\PY@tc{\PY@ul{%
    \PY@it{\PY@bf{\PY@ff{#1}}}}}}}
\def\PY#1#2{\PY@reset\PY@toks#1+\relax+\PY@do{#2}}

\expandafter\def\csname PY@tok@w\endcsname{\def\PY@tc##1{\textcolor[rgb]{0.73,0.73,0.73}{##1}}}
\expandafter\def\csname PY@tok@c\endcsname{\let\PY@it=\textit\def\PY@tc##1{\textcolor[rgb]{0.25,0.50,0.50}{##1}}}
\expandafter\def\csname PY@tok@cp\endcsname{\def\PY@tc##1{\textcolor[rgb]{0.74,0.48,0.00}{##1}}}
\expandafter\def\csname PY@tok@k\endcsname{\let\PY@bf=\textbf\def\PY@tc##1{\textcolor[rgb]{0.00,0.50,0.00}{##1}}}
\expandafter\def\csname PY@tok@kp\endcsname{\def\PY@tc##1{\textcolor[rgb]{0.00,0.50,0.00}{##1}}}
\expandafter\def\csname PY@tok@kt\endcsname{\def\PY@tc##1{\textcolor[rgb]{0.69,0.00,0.25}{##1}}}
\expandafter\def\csname PY@tok@o\endcsname{\def\PY@tc##1{\textcolor[rgb]{0.40,0.40,0.40}{##1}}}
\expandafter\def\csname PY@tok@ow\endcsname{\let\PY@bf=\textbf\def\PY@tc##1{\textcolor[rgb]{0.67,0.13,1.00}{##1}}}
\expandafter\def\csname PY@tok@nb\endcsname{\def\PY@tc##1{\textcolor[rgb]{0.00,0.50,0.00}{##1}}}
\expandafter\def\csname PY@tok@nf\endcsname{\def\PY@tc##1{\textcolor[rgb]{0.00,0.00,1.00}{##1}}}
\expandafter\def\csname PY@tok@nc\endcsname{\let\PY@bf=\textbf\def\PY@tc##1{\textcolor[rgb]{0.00,0.00,1.00}{##1}}}
\expandafter\def\csname PY@tok@nn\endcsname{\let\PY@bf=\textbf\def\PY@tc##1{\textcolor[rgb]{0.00,0.00,1.00}{##1}}}
\expandafter\def\csname PY@tok@ne\endcsname{\let\PY@bf=\textbf\def\PY@tc##1{\textcolor[rgb]{0.82,0.25,0.23}{##1}}}
\expandafter\def\csname PY@tok@nv\endcsname{\def\PY@tc##1{\textcolor[rgb]{0.10,0.09,0.49}{##1}}}
\expandafter\def\csname PY@tok@no\endcsname{\def\PY@tc##1{\textcolor[rgb]{0.53,0.00,0.00}{##1}}}
\expandafter\def\csname PY@tok@nl\endcsname{\def\PY@tc##1{\textcolor[rgb]{0.63,0.63,0.00}{##1}}}
\expandafter\def\csname PY@tok@ni\endcsname{\let\PY@bf=\textbf\def\PY@tc##1{\textcolor[rgb]{0.60,0.60,0.60}{##1}}}
\expandafter\def\csname PY@tok@na\endcsname{\def\PY@tc##1{\textcolor[rgb]{0.49,0.56,0.16}{##1}}}
\expandafter\def\csname PY@tok@nt\endcsname{\let\PY@bf=\textbf\def\PY@tc##1{\textcolor[rgb]{0.00,0.50,0.00}{##1}}}
\expandafter\def\csname PY@tok@nd\endcsname{\def\PY@tc##1{\textcolor[rgb]{0.67,0.13,1.00}{##1}}}
\expandafter\def\csname PY@tok@s\endcsname{\def\PY@tc##1{\textcolor[rgb]{0.73,0.13,0.13}{##1}}}
\expandafter\def\csname PY@tok@sd\endcsname{\let\PY@it=\textit\def\PY@tc##1{\textcolor[rgb]{0.73,0.13,0.13}{##1}}}
\expandafter\def\csname PY@tok@si\endcsname{\let\PY@bf=\textbf\def\PY@tc##1{\textcolor[rgb]{0.73,0.40,0.53}{##1}}}
\expandafter\def\csname PY@tok@se\endcsname{\let\PY@bf=\textbf\def\PY@tc##1{\textcolor[rgb]{0.73,0.40,0.13}{##1}}}
\expandafter\def\csname PY@tok@sr\endcsname{\def\PY@tc##1{\textcolor[rgb]{0.73,0.40,0.53}{##1}}}
\expandafter\def\csname PY@tok@ss\endcsname{\def\PY@tc##1{\textcolor[rgb]{0.10,0.09,0.49}{##1}}}
\expandafter\def\csname PY@tok@sx\endcsname{\def\PY@tc##1{\textcolor[rgb]{0.00,0.50,0.00}{##1}}}
\expandafter\def\csname PY@tok@m\endcsname{\def\PY@tc##1{\textcolor[rgb]{0.40,0.40,0.40}{##1}}}
\expandafter\def\csname PY@tok@gh\endcsname{\let\PY@bf=\textbf\def\PY@tc##1{\textcolor[rgb]{0.00,0.00,0.50}{##1}}}
\expandafter\def\csname PY@tok@gu\endcsname{\let\PY@bf=\textbf\def\PY@tc##1{\textcolor[rgb]{0.50,0.00,0.50}{##1}}}
\expandafter\def\csname PY@tok@gd\endcsname{\def\PY@tc##1{\textcolor[rgb]{0.63,0.00,0.00}{##1}}}
\expandafter\def\csname PY@tok@gi\endcsname{\def\PY@tc##1{\textcolor[rgb]{0.00,0.63,0.00}{##1}}}
\expandafter\def\csname PY@tok@gr\endcsname{\def\PY@tc##1{\textcolor[rgb]{1.00,0.00,0.00}{##1}}}
\expandafter\def\csname PY@tok@ge\endcsname{\let\PY@it=\textit}
\expandafter\def\csname PY@tok@gs\endcsname{\let\PY@bf=\textbf}
\expandafter\def\csname PY@tok@gp\endcsname{\let\PY@bf=\textbf\def\PY@tc##1{\textcolor[rgb]{0.00,0.00,0.50}{##1}}}
\expandafter\def\csname PY@tok@go\endcsname{\def\PY@tc##1{\textcolor[rgb]{0.53,0.53,0.53}{##1}}}
\expandafter\def\csname PY@tok@gt\endcsname{\def\PY@tc##1{\textcolor[rgb]{0.00,0.27,0.87}{##1}}}
\expandafter\def\csname PY@tok@err\endcsname{\def\PY@bc##1{\setlength{\fboxsep}{0pt}\fcolorbox[rgb]{1.00,0.00,0.00}{1,1,1}{\strut ##1}}}
\expandafter\def\csname PY@tok@kc\endcsname{\let\PY@bf=\textbf\def\PY@tc##1{\textcolor[rgb]{0.00,0.50,0.00}{##1}}}
\expandafter\def\csname PY@tok@kd\endcsname{\let\PY@bf=\textbf\def\PY@tc##1{\textcolor[rgb]{0.00,0.50,0.00}{##1}}}
\expandafter\def\csname PY@tok@kn\endcsname{\let\PY@bf=\textbf\def\PY@tc##1{\textcolor[rgb]{0.00,0.50,0.00}{##1}}}
\expandafter\def\csname PY@tok@kr\endcsname{\let\PY@bf=\textbf\def\PY@tc##1{\textcolor[rgb]{0.00,0.50,0.00}{##1}}}
\expandafter\def\csname PY@tok@bp\endcsname{\def\PY@tc##1{\textcolor[rgb]{0.00,0.50,0.00}{##1}}}
\expandafter\def\csname PY@tok@fm\endcsname{\def\PY@tc##1{\textcolor[rgb]{0.00,0.00,1.00}{##1}}}
\expandafter\def\csname PY@tok@vc\endcsname{\def\PY@tc##1{\textcolor[rgb]{0.10,0.09,0.49}{##1}}}
\expandafter\def\csname PY@tok@vg\endcsname{\def\PY@tc##1{\textcolor[rgb]{0.10,0.09,0.49}{##1}}}
\expandafter\def\csname PY@tok@vi\endcsname{\def\PY@tc##1{\textcolor[rgb]{0.10,0.09,0.49}{##1}}}
\expandafter\def\csname PY@tok@vm\endcsname{\def\PY@tc##1{\textcolor[rgb]{0.10,0.09,0.49}{##1}}}
\expandafter\def\csname PY@tok@sa\endcsname{\def\PY@tc##1{\textcolor[rgb]{0.73,0.13,0.13}{##1}}}
\expandafter\def\csname PY@tok@sb\endcsname{\def\PY@tc##1{\textcolor[rgb]{0.73,0.13,0.13}{##1}}}
\expandafter\def\csname PY@tok@sc\endcsname{\def\PY@tc##1{\textcolor[rgb]{0.73,0.13,0.13}{##1}}}
\expandafter\def\csname PY@tok@dl\endcsname{\def\PY@tc##1{\textcolor[rgb]{0.73,0.13,0.13}{##1}}}
\expandafter\def\csname PY@tok@s2\endcsname{\def\PY@tc##1{\textcolor[rgb]{0.73,0.13,0.13}{##1}}}
\expandafter\def\csname PY@tok@sh\endcsname{\def\PY@tc##1{\textcolor[rgb]{0.73,0.13,0.13}{##1}}}
\expandafter\def\csname PY@tok@s1\endcsname{\def\PY@tc##1{\textcolor[rgb]{0.73,0.13,0.13}{##1}}}
\expandafter\def\csname PY@tok@mb\endcsname{\def\PY@tc##1{\textcolor[rgb]{0.40,0.40,0.40}{##1}}}
\expandafter\def\csname PY@tok@mf\endcsname{\def\PY@tc##1{\textcolor[rgb]{0.40,0.40,0.40}{##1}}}
\expandafter\def\csname PY@tok@mh\endcsname{\def\PY@tc##1{\textcolor[rgb]{0.40,0.40,0.40}{##1}}}
\expandafter\def\csname PY@tok@mi\endcsname{\def\PY@tc##1{\textcolor[rgb]{0.40,0.40,0.40}{##1}}}
\expandafter\def\csname PY@tok@il\endcsname{\def\PY@tc##1{\textcolor[rgb]{0.40,0.40,0.40}{##1}}}
\expandafter\def\csname PY@tok@mo\endcsname{\def\PY@tc##1{\textcolor[rgb]{0.40,0.40,0.40}{##1}}}
\expandafter\def\csname PY@tok@ch\endcsname{\let\PY@it=\textit\def\PY@tc##1{\textcolor[rgb]{0.25,0.50,0.50}{##1}}}
\expandafter\def\csname PY@tok@cm\endcsname{\let\PY@it=\textit\def\PY@tc##1{\textcolor[rgb]{0.25,0.50,0.50}{##1}}}
\expandafter\def\csname PY@tok@cpf\endcsname{\let\PY@it=\textit\def\PY@tc##1{\textcolor[rgb]{0.25,0.50,0.50}{##1}}}
\expandafter\def\csname PY@tok@c1\endcsname{\let\PY@it=\textit\def\PY@tc##1{\textcolor[rgb]{0.25,0.50,0.50}{##1}}}
\expandafter\def\csname PY@tok@cs\endcsname{\let\PY@it=\textit\def\PY@tc##1{\textcolor[rgb]{0.25,0.50,0.50}{##1}}}

\def\PYZbs{\char`\\}
\def\PYZus{\char`\_}
\def\PYZob{\char`\{}
\def\PYZcb{\char`\}}
\def\PYZca{\char`\^}
\def\PYZam{\char`\&}
\def\PYZlt{\char`\<}
\def\PYZgt{\char`\>}
\def\PYZsh{\char`\#}
\def\PYZpc{\char`\%}
\def\PYZdl{\char`\$}
\def\PYZhy{\char`\-}
\def\PYZsq{\char`\'}
\def\PYZdq{\char`\"}
\def\PYZti{\char`\~}
% for compatibility with earlier versions
\def\PYZat{@}
\def\PYZlb{[}
\def\PYZrb{]}
\makeatother


    % For linebreaks inside Verbatim environment from package fancyvrb. 
    \makeatletter
        \newbox\Wrappedcontinuationbox 
        \newbox\Wrappedvisiblespacebox 
        \newcommand*\Wrappedvisiblespace {\textcolor{red}{\textvisiblespace}} 
        \newcommand*\Wrappedcontinuationsymbol {\textcolor{red}{\llap{\tiny$\m@th\hookrightarrow$}}} 
        \newcommand*\Wrappedcontinuationindent {3ex } 
        \newcommand*\Wrappedafterbreak {\kern\Wrappedcontinuationindent\copy\Wrappedcontinuationbox} 
        % Take advantage of the already applied Pygments mark-up to insert 
        % potential linebreaks for TeX processing. 
        %        {, <, #, %, $, ' and ": go to next line. 
        %        _, }, ^, &, >, - and ~: stay at end of broken line. 
        % Use of \textquotesingle for straight quote. 
        \newcommand*\Wrappedbreaksatspecials {% 
            \def\PYGZus{\discretionary{\char`\_}{\Wrappedafterbreak}{\char`\_}}% 
            \def\PYGZob{\discretionary{}{\Wrappedafterbreak\char`\{}{\char`\{}}% 
            \def\PYGZcb{\discretionary{\char`\}}{\Wrappedafterbreak}{\char`\}}}% 
            \def\PYGZca{\discretionary{\char`\^}{\Wrappedafterbreak}{\char`\^}}% 
            \def\PYGZam{\discretionary{\char`\&}{\Wrappedafterbreak}{\char`\&}}% 
            \def\PYGZlt{\discretionary{}{\Wrappedafterbreak\char`\<}{\char`\<}}% 
            \def\PYGZgt{\discretionary{\char`\>}{\Wrappedafterbreak}{\char`\>}}% 
            \def\PYGZsh{\discretionary{}{\Wrappedafterbreak\char`\#}{\char`\#}}% 
            \def\PYGZpc{\discretionary{}{\Wrappedafterbreak\char`\%}{\char`\%}}% 
            \def\PYGZdl{\discretionary{}{\Wrappedafterbreak\char`\$}{\char`\$}}% 
            \def\PYGZhy{\discretionary{\char`\-}{\Wrappedafterbreak}{\char`\-}}% 
            \def\PYGZsq{\discretionary{}{\Wrappedafterbreak\textquotesingle}{\textquotesingle}}% 
            \def\PYGZdq{\discretionary{}{\Wrappedafterbreak\char`\"}{\char`\"}}% 
            \def\PYGZti{\discretionary{\char`\~}{\Wrappedafterbreak}{\char`\~}}% 
        } 
        % Some characters . , ; ? ! / are not pygmentized. 
        % This macro makes them "active" and they will insert potential linebreaks 
        \newcommand*\Wrappedbreaksatpunct {% 
            \lccode`\~`\.\lowercase{\def~}{\discretionary{\hbox{\char`\.}}{\Wrappedafterbreak}{\hbox{\char`\.}}}% 
            \lccode`\~`\,\lowercase{\def~}{\discretionary{\hbox{\char`\,}}{\Wrappedafterbreak}{\hbox{\char`\,}}}% 
            \lccode`\~`\;\lowercase{\def~}{\discretionary{\hbox{\char`\;}}{\Wrappedafterbreak}{\hbox{\char`\;}}}% 
            \lccode`\~`\:\lowercase{\def~}{\discretionary{\hbox{\char`\:}}{\Wrappedafterbreak}{\hbox{\char`\:}}}% 
            \lccode`\~`\?\lowercase{\def~}{\discretionary{\hbox{\char`\?}}{\Wrappedafterbreak}{\hbox{\char`\?}}}% 
            \lccode`\~`\!\lowercase{\def~}{\discretionary{\hbox{\char`\!}}{\Wrappedafterbreak}{\hbox{\char`\!}}}% 
            \lccode`\~`\/\lowercase{\def~}{\discretionary{\hbox{\char`\/}}{\Wrappedafterbreak}{\hbox{\char`\/}}}% 
            \catcode`\.\active
            \catcode`\,\active 
            \catcode`\;\active
            \catcode`\:\active
            \catcode`\?\active
            \catcode`\!\active
            \catcode`\/\active 
            \lccode`\~`\~ 	
        }
    \makeatother

    \let\OriginalVerbatim=\Verbatim
    \makeatletter
    \renewcommand{\Verbatim}[1][1]{%
        %\parskip\z@skip
        \sbox\Wrappedcontinuationbox {\Wrappedcontinuationsymbol}%
        \sbox\Wrappedvisiblespacebox {\FV@SetupFont\Wrappedvisiblespace}%
        \def\FancyVerbFormatLine ##1{\hsize\linewidth
            \vtop{\raggedright\hyphenpenalty\z@\exhyphenpenalty\z@
                \doublehyphendemerits\z@\finalhyphendemerits\z@
                \strut ##1\strut}%
        }%
        % If the linebreak is at a space, the latter will be displayed as visible
        % space at end of first line, and a continuation symbol starts next line.
        % Stretch/shrink are however usually zero for typewriter font.
        \def\FV@Space {%
            \nobreak\hskip\z@ plus\fontdimen3\font minus\fontdimen4\font
            \discretionary{\copy\Wrappedvisiblespacebox}{\Wrappedafterbreak}
            {\kern\fontdimen2\font}%
        }%
        
        % Allow breaks at special characters using \PYG... macros.
        \Wrappedbreaksatspecials
        % Breaks at punctuation characters . , ; ? ! and / need catcode=\active 	
        \OriginalVerbatim[#1,codes*=\Wrappedbreaksatpunct]%
    }
    \makeatother

    % Exact colors from NB
    \definecolor{incolor}{HTML}{303F9F}
    \definecolor{outcolor}{HTML}{D84315}
    \definecolor{cellborder}{HTML}{CFCFCF}
    \definecolor{cellbackground}{HTML}{F7F7F7}
    
    % prompt
    \makeatletter
    \newcommand{\boxspacing}{\kern\kvtcb@left@rule\kern\kvtcb@boxsep}
    \makeatother
    \newcommand{\prompt}[4]{
        \ttfamily\llap{{\color{#2}[#3]:\hspace{3pt}#4}}\vspace{-\baselineskip}
    }
    

    
    % Prevent overflowing lines due to hard-to-break entities
    \sloppy 
    % Setup hyperref package
    \hypersetup{
      breaklinks=true,  % so long urls are correctly broken across lines
      colorlinks=true,
      urlcolor=urlcolor,
      linkcolor=linkcolor,
      citecolor=citecolor,
      }
    % Slightly bigger margins than the latex defaults
    
    \geometry{verbose,tmargin=1in,bmargin=1in,lmargin=1in,rmargin=1in}
    
    

\begin{document}
    
    \maketitle
    
    

    
    \begin{tcolorbox}[breakable, size=fbox, boxrule=1pt, pad at break*=1mm,colback=cellbackground, colframe=cellborder]
\prompt{In}{incolor}{3}{\boxspacing}
\begin{Verbatim}[commandchars=\\\{\}]
\PY{n+nv}{LoadPackage}\PY{p}{(} \PY{l+s}{\PYZdq{}DerivedCategories\PYZdq{}} \PY{p}{)}\PY{o}{;}
\PY{n+nv}{ENABLE\PYZus{}COLORS} \PY{o}{:=} \PY{n+no}{true}\PY{o}{;}\PY{o}{;}
\PY{n+nv}{SizeScreen}\PY{p}{(}\PY{p}{[}\PY{n+nv}{1000}\PY{o}{,} \PY{n+nv}{1000}\PY{p}{]}\PY{p}{)}\PY{o}{;}\PY{o}{;}
\end{Verbatim}
\end{tcolorbox}

    \begin{Verbatim}[commandchars=\\\{\}]
\#I  method installed for DeterminantMatrix matches more than one declaration
\#I  method installed for IdentityMatrix matches more than one declaration
\#I  method installed for RowsOfMatrix matches more than one declaration
\#I  method installed for RowsOfMatrix matches more than one declaration
\#I  method installed for RowsOfMatrix matches more than one declaration
\#I  method installed for UnderlyingField matches more than one declaration
\#I  method installed for + matches more than one declaration
\#I  method installed for IsZero matches more than one declaration
\#I  method installed for InverseMutable matches more than one declaration
\#I  method installed for IsZero matches more than one declaration
\#I  method installed for + matches more than one declaration
\#I  method installed for InverseMutable matches more than one declaration
\#I  method installed for IsZero matches more than one declaration
\#I  method installed for + matches more than one declaration
\#I  method installed for InverseMutable matches more than one declaration
    \end{Verbatim}

            \begin{tcolorbox}[breakable, size=fbox, boxrule=.5pt, pad at break*=1mm, opacityfill=0]
\prompt{Out}{outcolor}{3}{\boxspacing}
\begin{Verbatim}[commandchars=\\\{\}]
true
\end{Verbatim}
\end{tcolorbox}
        
    \begin{tcolorbox}[breakable, size=fbox, boxrule=1pt, pad at break*=1mm,colback=cellbackground, colframe=cellborder]
\prompt{In}{incolor}{4}{\boxspacing}
\begin{Verbatim}[commandchars=\\\{\}]
\PY{n+nv}{k} \PY{o}{:=} \PY{n+nv}{HomalgFieldOfRationals}\PY{p}{(} \PY{p}{)}\PY{o}{;}
\end{Verbatim}
\end{tcolorbox}

            \begin{tcolorbox}[breakable, size=fbox, boxrule=.5pt, pad at break*=1mm, opacityfill=0]
\prompt{Out}{outcolor}{4}{\boxspacing}
\begin{Verbatim}[commandchars=\\\{\}]
<field in characteristic 0>
\end{Verbatim}
\end{tcolorbox}
        
    \begin{figure}
\centering
\includegraphics{attachment:1.png}
\caption{1.png}
\end{figure}

    \begin{tcolorbox}[breakable, size=fbox, boxrule=1pt, pad at break*=1mm,colback=cellbackground, colframe=cellborder]
\prompt{In}{incolor}{5}{\boxspacing}
\begin{Verbatim}[commandchars=\\\{\}]
\PY{n+nv}{vertices} \PY{o}{:=} \PY{p}{[} \PY{l+s}{\PYZdq{}A\PYZdq{}}\PY{o}{,} \PY{l+s}{\PYZdq{}B\PYZdq{}}\PY{o}{,} \PY{l+s}{\PYZdq{}C\PYZdq{}}\PY{o}{,} \PY{l+s}{\PYZdq{}D\PYZdq{}} \PY{p}{]}\PY{o}{;}
\end{Verbatim}
\end{tcolorbox}

            \begin{tcolorbox}[breakable, size=fbox, boxrule=.5pt, pad at break*=1mm, opacityfill=0]
\prompt{Out}{outcolor}{5}{\boxspacing}
\begin{Verbatim}[commandchars=\\\{\}]
[ "A", "B", "C", "D" ]
\end{Verbatim}
\end{tcolorbox}
        
    \begin{tcolorbox}[breakable, size=fbox, boxrule=1pt, pad at break*=1mm,colback=cellbackground, colframe=cellborder]
\prompt{In}{incolor}{8}{\boxspacing}
\begin{Verbatim}[commandchars=\\\{\}]
\PY{n+nv}{arrows} \PY{o}{:=} \PY{p}{[} \PY{l+s}{\PYZdq{}r\PYZdq{}}\PY{o}{,} \PY{l+s}{\PYZdq{}s\PYZdq{}}\PY{o}{,} \PY{l+s}{\PYZdq{}f\PYZus{}0\PYZdq{}}\PY{o}{,} \PY{l+s}{\PYZdq{}f\PYZus{}1\PYZdq{}}\PY{o}{,} \PY{l+s}{\PYZdq{}g\PYZus{}0\PYZdq{}}\PY{o}{,} \PY{l+s}{\PYZdq{}g\PYZus{}1\PYZdq{}}\PY{o}{,} \PY{l+s}{\PYZdq{}u\PYZdq{}}\PY{o}{,} \PY{l+s}{\PYZdq{}v\PYZdq{}} \PY{p}{]}\PY{o}{;}
\PY{n+nv}{sources} \PY{o}{:=} \PY{p}{[} \PY{n+nv}{1}\PY{o}{,} \PY{n+nv}{2}\PY{o}{,} \PY{n+nv}{1}\PY{o}{,} \PY{n+nv}{3}\PY{o}{,} \PY{n+nv}{1}\PY{o}{,} \PY{n+nv}{3}\PY{o}{,} \PY{n+nv}{3}\PY{o}{,} \PY{n+nv}{3} \PY{p}{]}\PY{o}{;}\PY{o}{;}
\PY{n+nv}{targets}  \PY{o}{:=} \PY{p}{[} \PY{n+nv}{3}\PY{o}{,} \PY{n+nv}{4}\PY{o}{,} \PY{n+nv}{2}\PY{o}{,} \PY{n+nv}{4}\PY{o}{,} \PY{n+nv}{2}\PY{o}{,} \PY{n+nv}{4}\PY{o}{,} \PY{n+nv}{2}\PY{o}{,} \PY{n+nv}{2} \PY{p}{]}\PY{o}{;}\PY{o}{;}
\end{Verbatim}
\end{tcolorbox}

            \begin{tcolorbox}[breakable, size=fbox, boxrule=.5pt, pad at break*=1mm, opacityfill=0]
\prompt{Out}{outcolor}{8}{\boxspacing}
\begin{Verbatim}[commandchars=\\\{\}]
[ "r", "s", "f\_0", "f\_1", "g\_0", "g\_1", "u", "v" ]
\end{Verbatim}
\end{tcolorbox}
        
    \begin{tcolorbox}[breakable, size=fbox, boxrule=1pt, pad at break*=1mm,colback=cellbackground, colframe=cellborder]
\prompt{In}{incolor}{10}{\boxspacing}
\begin{Verbatim}[commandchars=\\\{\}]
\PY{n+nv}{Q} \PY{o}{:=} \PY{n+nv}{RightQuiver}\PY{p}{(} \PY{l+s}{\PYZdq{}quiver\PYZdq{}}\PY{o}{,} \PY{n+nv}{vertices}\PY{o}{,} \PY{n+nv}{arrows}\PY{o}{,} \PY{n+nv}{sources}\PY{o}{,} \PY{n+nv}{targets} \PY{p}{)}\PY{o}{;}\PY{o}{;}
\PY{n+nv}{ViewObj}\PY{p}{(} \PY{n+nv}{Q} \PY{p}{)}\PY{o}{;}
\end{Verbatim}
\end{tcolorbox}

    \begin{Verbatim}[commandchars=\\\{\}]
quiver(A,B,C,D)[r:A->C,s:B->D,f\_0:A->B,f\_1:C->D,g\_0:A->B,g\_1:C->D,u:C->B,v:C->B]
    \end{Verbatim}

    \begin{tcolorbox}[breakable, size=fbox, boxrule=1pt, pad at break*=1mm,colback=cellbackground, colframe=cellborder]
\prompt{In}{incolor}{12}{\boxspacing}
\begin{Verbatim}[commandchars=\\\{\}]
\PY{n+nv}{kQ} \PY{o}{:=} \PY{n+nv}{PathAlgebra}\PY{p}{(} \PY{n+nv}{k}\PY{o}{,} \PY{n+nv}{Q} \PY{p}{)}\PY{o}{;}\PY{o}{;}
\PY{n+nv}{ViewObj}\PY{p}{(} \PY{n+nv}{kQ} \PY{p}{)}\PY{o}{;}
\end{Verbatim}
\end{tcolorbox}

    \begin{Verbatim}[commandchars=\\\{\}]
Q * quiver
    \end{Verbatim}

    \begin{tcolorbox}[breakable, size=fbox, boxrule=1pt, pad at break*=1mm,colback=cellbackground, colframe=cellborder]
\prompt{In}{incolor}{13}{\boxspacing}
\begin{Verbatim}[commandchars=\\\{\}]
\PY{n+nv}{I} \PY{o}{:=} \PY{p}{[}
      \PY{n+nv}{kQ}\PY{o}{.}\PY{n+nv}{r} \PY{o}{*} \PY{n+nv}{kQ}\PY{o}{.}\PY{n+nv}{f\PYZus{}1} \PY{o}{\PYZhy{}} \PY{n+nv}{kQ}\PY{o}{.}\PY{n+nv}{f\PYZus{}0} \PY{o}{*} \PY{n+nv}{kQ}\PY{o}{.}\PY{n+nv}{s}\PY{o}{,}
      \PY{n+nv}{kQ}\PY{o}{.}\PY{n+nv}{r} \PY{o}{*} \PY{n+nv}{kQ}\PY{o}{.}\PY{n+nv}{g\PYZus{}1} \PY{o}{\PYZhy{}} \PY{n+nv}{kQ}\PY{o}{.}\PY{n+nv}{g\PYZus{}0} \PY{o}{*} \PY{n+nv}{kQ}\PY{o}{.}\PY{n+nv}{s}\PY{o}{,}
      \PY{n+nv}{kQ}\PY{o}{.}\PY{n+nv}{f\PYZus{}1} \PY{o}{\PYZhy{}} \PY{n+nv}{kQ}\PY{o}{.}\PY{n+nv}{g\PYZus{}1}\PY{o}{\PYZhy{}} \PY{n+nv}{kQ}\PY{o}{.}\PY{n+nv}{v} \PY{o}{*} \PY{n+nv}{kQ}\PY{o}{.}\PY{n+nv}{s}\PY{o}{,}
      \PY{n+nv}{kQ}\PY{o}{.}\PY{n+nv}{f\PYZus{}0} \PY{o}{\PYZhy{}} \PY{n+nv}{kQ}\PY{o}{.}\PY{n+nv}{g\PYZus{}0} \PY{o}{\PYZhy{}} \PY{n+nv}{kQ}\PY{o}{.}\PY{n+nv}{r} \PY{o}{*} \PY{n+nv}{kQ}\PY{o}{.}\PY{n+nv}{u}
    \PY{p}{]}\PY{o}{;}
\end{Verbatim}
\end{tcolorbox}

            \begin{tcolorbox}[breakable, size=fbox, boxrule=.5pt, pad at break*=1mm, opacityfill=0]
\prompt{Out}{outcolor}{13}{\boxspacing}
\begin{Verbatim}[commandchars=\\\{\}]
[ -1*(f\_0*s) + 1*(r*f\_1), -1*(g\_0*s) + 1*(r*g\_1), -1*(v*s) - 1*(g\_1) + 1*(f\_1),
-1*(r*u) - 1*(g\_0) + 1*(f\_0) ]
\end{Verbatim}
\end{tcolorbox}
        
    \begin{tcolorbox}[breakable, size=fbox, boxrule=1pt, pad at break*=1mm,colback=cellbackground, colframe=cellborder]
\prompt{In}{incolor}{16}{\boxspacing}
\begin{Verbatim}[commandchars=\\\{\}]
\PY{n+nv}{kQ\PYZus{}mod\PYZus{}I} \PY{o}{:=} \PY{n+nv}{kQ} \PY{o}{/} \PY{n+nv}{I}\PY{o}{;}\PY{o}{;}
\PY{n+nv}{SetName}\PY{p}{(} \PY{n+nv}{kQ\PYZus{}mod\PYZus{}I}\PY{o}{,} \PY{l+s}{\PYZdq{}kQ/I\PYZdq{}} \PY{p}{)}\PY{o}{;}
\PY{n+nv}{ViewObj}\PY{p}{(} \PY{n+nv}{kQ\PYZus{}mod\PYZus{}I} \PY{p}{)}\PY{o}{;}
\end{Verbatim}
\end{tcolorbox}

    \begin{Verbatim}[commandchars=\\\{\}]
kQ/I
    \end{Verbatim}

    \begin{tcolorbox}[breakable, size=fbox, boxrule=1pt, pad at break*=1mm,colback=cellbackground, colframe=cellborder]
\prompt{In}{incolor}{17}{\boxspacing}
\begin{Verbatim}[commandchars=\\\{\}]
\PY{n+nv}{Dimension}\PY{p}{(} \PY{n+nv}{kQ\PYZus{}mod\PYZus{}I} \PY{p}{)}\PY{o}{;}
\end{Verbatim}
\end{tcolorbox}

            \begin{tcolorbox}[breakable, size=fbox, boxrule=.5pt, pad at break*=1mm, opacityfill=0]
\prompt{Out}{outcolor}{17}{\boxspacing}
\begin{Verbatim}[commandchars=\\\{\}]
16
\end{Verbatim}
\end{tcolorbox}
        
    \begin{tcolorbox}[breakable, size=fbox, boxrule=1pt, pad at break*=1mm,colback=cellbackground, colframe=cellborder]
\prompt{In}{incolor}{18}{\boxspacing}
\begin{Verbatim}[commandchars=\\\{\}]
\PY{n+nv}{KnownAttributesOfObject}\PY{p}{(} \PY{n+nv}{kQ\PYZus{}mod\PYZus{}I} \PY{p}{)}\PY{o}{;}
\end{Verbatim}
\end{tcolorbox}

            \begin{tcolorbox}[breakable, size=fbox, boxrule=.5pt, pad at break*=1mm, opacityfill=0]
\prompt{Out}{outcolor}{18}{\boxspacing}
\begin{Verbatim}[commandchars=\\\{\}]
[ "Name", "LeftActingDomain", "MultiplicativeNeutralElement", "Dimension",
"GeneratorsOfLeftOperatorAdditiveGroup", "Basis", "CanonicalBasis",
"QuiverOfAlgebra", "IdealOfQuotient", "PathAlgebra", "Hash" ]
\end{Verbatim}
\end{tcolorbox}
        
    \begin{tcolorbox}[breakable, size=fbox, boxrule=1pt, pad at break*=1mm,colback=cellbackground, colframe=cellborder]
\prompt{In}{incolor}{19}{\boxspacing}
\begin{Verbatim}[commandchars=\\\{\}]
\PY{n+nv}{oid} \PY{o}{:=} \PY{n+nv}{Algebroid}\PY{p}{(} \PY{n+nv}{kQ\PYZus{}mod\PYZus{}I} \PY{p}{)}\PY{o}{;}
\end{Verbatim}
\end{tcolorbox}

            \begin{tcolorbox}[breakable, size=fbox, boxrule=.5pt, pad at break*=1mm, opacityfill=0]
\prompt{Out}{outcolor}{19}{\boxspacing}
\begin{Verbatim}[commandchars=\\\{\}]
\textcolor{ansi-green}{Algebroid( }kQ/I\textcolor{ansi-green}{ )}
\end{Verbatim}
\end{tcolorbox}
        
    \begin{tcolorbox}[breakable, size=fbox, boxrule=1pt, pad at break*=1mm,colback=cellbackground, colframe=cellborder]
\prompt{In}{incolor}{21}{\boxspacing}
\begin{Verbatim}[commandchars=\\\{\}]
\PY{n+nv}{objs} \PY{o}{:=} \PY{n+nv}{SetOfObjects}\PY{p}{(} \PY{n+nv}{oid} \PY{p}{)}\PY{o}{;}\PY{o}{;}
\PY{n+nv}{Size}\PY{p}{(} \PY{n+nv}{objs} \PY{p}{)}\PY{o}{;}
\end{Verbatim}
\end{tcolorbox}

            \begin{tcolorbox}[breakable, size=fbox, boxrule=.5pt, pad at break*=1mm, opacityfill=0]
\prompt{Out}{outcolor}{21}{\boxspacing}
\begin{Verbatim}[commandchars=\\\{\}]
4
\end{Verbatim}
\end{tcolorbox}
        
    \begin{tcolorbox}[breakable, size=fbox, boxrule=1pt, pad at break*=1mm,colback=cellbackground, colframe=cellborder]
\prompt{In}{incolor}{22}{\boxspacing}
\begin{Verbatim}[commandchars=\\\{\}]
\PY{n+nv}{objs}\PY{p}{[} \PY{n+nv}{1} \PY{p}{]}\PY{o}{;}
\end{Verbatim}
\end{tcolorbox}

            \begin{tcolorbox}[breakable, size=fbox, boxrule=.5pt, pad at break*=1mm, opacityfill=0]
\prompt{Out}{outcolor}{22}{\boxspacing}
\begin{Verbatim}[commandchars=\\\{\}]
<An object in \textcolor{ansi-green}{Algebroid( }kQ/I\textcolor{ansi-green}{ )}>
\end{Verbatim}
\end{tcolorbox}
        
    \begin{tcolorbox}[breakable, size=fbox, boxrule=1pt, pad at break*=1mm,colback=cellbackground, colframe=cellborder]
\prompt{In}{incolor}{23}{\boxspacing}
\begin{Verbatim}[commandchars=\\\{\}]
\PY{n+nv}{ViewObj}\PY{p}{(} \PY{n+nv}{objs}\PY{p}{[} \PY{n+nv}{1} \PY{p}{]} \PY{p}{)}\PY{o}{;}
\end{Verbatim}
\end{tcolorbox}

    \begin{Verbatim}[commandchars=\\\{\}]
<(A)>
    \end{Verbatim}

    \begin{tcolorbox}[breakable, size=fbox, boxrule=1pt, pad at break*=1mm,colback=cellbackground, colframe=cellborder]
\prompt{In}{incolor}{24}{\boxspacing}
\begin{Verbatim}[commandchars=\\\{\}]
\PY{n+nv}{objs}\PY{p}{[} \PY{n+nv}{1} \PY{p}{]} \PY{o}{=} \PY{n+nv}{oid}\PY{o}{.}\PY{p}{(} \PY{l+s}{\PYZdq{}A\PYZdq{}} \PY{p}{)}\PY{o}{;}
\end{Verbatim}
\end{tcolorbox}

            \begin{tcolorbox}[breakable, size=fbox, boxrule=.5pt, pad at break*=1mm, opacityfill=0]
\prompt{Out}{outcolor}{24}{\boxspacing}
\begin{Verbatim}[commandchars=\\\{\}]
true
\end{Verbatim}
\end{tcolorbox}
        
    \begin{tcolorbox}[breakable, size=fbox, boxrule=1pt, pad at break*=1mm,colback=cellbackground, colframe=cellborder]
\prompt{In}{incolor}{26}{\boxspacing}
\begin{Verbatim}[commandchars=\\\{\}]
\PY{n+nv}{mors} \PY{o}{:=} \PY{n+nv}{SetOfGeneratingMorphisms}\PY{p}{(} \PY{n+nv}{oid} \PY{p}{)}\PY{o}{;}\PY{o}{;}
\PY{n+nv}{Size}\PY{p}{(} \PY{n+nv}{mors} \PY{p}{)}\PY{o}{;}
\end{Verbatim}
\end{tcolorbox}

            \begin{tcolorbox}[breakable, size=fbox, boxrule=.5pt, pad at break*=1mm, opacityfill=0]
\prompt{Out}{outcolor}{26}{\boxspacing}
\begin{Verbatim}[commandchars=\\\{\}]
8
\end{Verbatim}
\end{tcolorbox}
        
    \begin{tcolorbox}[breakable, size=fbox, boxrule=1pt, pad at break*=1mm,colback=cellbackground, colframe=cellborder]
\prompt{In}{incolor}{27}{\boxspacing}
\begin{Verbatim}[commandchars=\\\{\}]
\PY{n+nv}{mors}\PY{p}{[} \PY{n+nv}{1} \PY{p}{]}\PY{o}{;}
\end{Verbatim}
\end{tcolorbox}

            \begin{tcolorbox}[breakable, size=fbox, boxrule=.5pt, pad at break*=1mm, opacityfill=0]
\prompt{Out}{outcolor}{27}{\boxspacing}
\begin{Verbatim}[commandchars=\\\{\}]
<A morphism in \textcolor{ansi-green}{Algebroid( }kQ/I\textcolor{ansi-green}{ )}>
\end{Verbatim}
\end{tcolorbox}
        
    \begin{tcolorbox}[breakable, size=fbox, boxrule=1pt, pad at break*=1mm,colback=cellbackground, colframe=cellborder]
\prompt{In}{incolor}{28}{\boxspacing}
\begin{Verbatim}[commandchars=\\\{\}]
\PY{n+nv}{ViewObj}\PY{p}{(} \PY{n+nv}{mors}\PY{p}{[} \PY{n+nv}{1} \PY{p}{]} \PY{p}{)}\PY{o}{;}
\end{Verbatim}
\end{tcolorbox}

    \begin{Verbatim}[commandchars=\\\{\}]
(A)-[\{ 1*(r) \}]->(C)
    \end{Verbatim}

    \begin{tcolorbox}[breakable, size=fbox, boxrule=1pt, pad at break*=1mm,colback=cellbackground, colframe=cellborder]
\prompt{In}{incolor}{29}{\boxspacing}
\begin{Verbatim}[commandchars=\\\{\}]
\PY{n+nv}{mors}\PY{p}{[} \PY{n+nv}{1} \PY{p}{]} \PY{o}{=} \PY{n+nv}{oid}\PY{o}{.}\PY{p}{(} \PY{l+s}{\PYZdq{}r\PYZdq{}} \PY{p}{)}\PY{o}{;}
\end{Verbatim}
\end{tcolorbox}

            \begin{tcolorbox}[breakable, size=fbox, boxrule=.5pt, pad at break*=1mm, opacityfill=0]
\prompt{Out}{outcolor}{29}{\boxspacing}
\begin{Verbatim}[commandchars=\\\{\}]
true
\end{Verbatim}
\end{tcolorbox}
        
    \begin{tcolorbox}[breakable, size=fbox, boxrule=1pt, pad at break*=1mm,colback=cellbackground, colframe=cellborder]
\prompt{In}{incolor}{30}{\boxspacing}
\begin{Verbatim}[commandchars=\\\{\}]
\PY{n+nv}{r\PYZus{}f1} \PY{o}{:=} \PY{n+nv}{PreCompose}\PY{p}{(} \PY{n+nv}{oid}\PY{o}{.}\PY{p}{(} \PY{l+s}{\PYZdq{}r\PYZdq{}} \PY{p}{)}\PY{o}{,} \PY{n+nv}{oid}\PY{o}{.}\PY{p}{(} \PY{l+s}{\PYZdq{}f\PYZus{}1\PYZdq{}} \PY{p}{)} \PY{p}{)}\PY{o}{;}
\end{Verbatim}
\end{tcolorbox}

            \begin{tcolorbox}[breakable, size=fbox, boxrule=.5pt, pad at break*=1mm, opacityfill=0]
\prompt{Out}{outcolor}{30}{\boxspacing}
\begin{Verbatim}[commandchars=\\\{\}]
<A morphism in \textcolor{ansi-green}{Algebroid( }kQ/I\textcolor{ansi-green}{ )}>
\end{Verbatim}
\end{tcolorbox}
        
    \begin{tcolorbox}[breakable, size=fbox, boxrule=1pt, pad at break*=1mm,colback=cellbackground, colframe=cellborder]
\prompt{In}{incolor}{31}{\boxspacing}
\begin{Verbatim}[commandchars=\\\{\}]
\PY{n+nv}{ViewObj}\PY{p}{(} \PY{n+nv}{r\PYZus{}f1} \PY{p}{)}\PY{o}{;}
\end{Verbatim}
\end{tcolorbox}

    \begin{Verbatim}[commandchars=\\\{\}]
(A)-[\{ 1*(r*f\_1) \}]->(D)
    \end{Verbatim}

    \begin{tcolorbox}[breakable, size=fbox, boxrule=1pt, pad at break*=1mm,colback=cellbackground, colframe=cellborder]
\prompt{In}{incolor}{32}{\boxspacing}
\begin{Verbatim}[commandchars=\\\{\}]
\PY{n+nv}{PreCompose}\PY{p}{(} \PY{n+nv}{oid}\PY{o}{.}\PY{p}{(} \PY{l+s}{\PYZdq{}r\PYZdq{}} \PY{p}{)}\PY{o}{,} \PY{n+nv}{oid}\PY{o}{.}\PY{p}{(} \PY{l+s}{\PYZdq{}f\PYZus{}1\PYZdq{}} \PY{p}{)} \PY{p}{)} \PY{o}{=} \PY{n+nv}{PreCompose}\PY{p}{(} \PY{n+nv}{oid}\PY{o}{.}\PY{p}{(} \PY{l+s}{\PYZdq{}f\PYZus{}0\PYZdq{}} \PY{p}{)}\PY{o}{,} \PY{n+nv}{oid}\PY{o}{.}\PY{p}{(} \PY{l+s}{\PYZdq{}s\PYZdq{}} \PY{p}{)} \PY{p}{)}\PY{o}{;}
\end{Verbatim}
\end{tcolorbox}

            \begin{tcolorbox}[breakable, size=fbox, boxrule=.5pt, pad at break*=1mm, opacityfill=0]
\prompt{Out}{outcolor}{32}{\boxspacing}
\begin{Verbatim}[commandchars=\\\{\}]
true
\end{Verbatim}
\end{tcolorbox}
        
    \begin{tcolorbox}[breakable, size=fbox, boxrule=1pt, pad at break*=1mm,colback=cellbackground, colframe=cellborder]
\prompt{In}{incolor}{33}{\boxspacing}
\begin{Verbatim}[commandchars=\\\{\}]
\PY{n+nv}{oplus} \PY{o}{:=} \PY{n+nv}{AdditiveClosure}\PY{p}{(} \PY{n+nv}{oid} \PY{p}{)}\PY{o}{;}
\end{Verbatim}
\end{tcolorbox}

            \begin{tcolorbox}[breakable, size=fbox, boxrule=.5pt, pad at break*=1mm, opacityfill=0]
\prompt{Out}{outcolor}{33}{\boxspacing}
\begin{Verbatim}[commandchars=\\\{\}]
Additive closure( \textcolor{ansi-green}{Algebroid( }kQ/I\textcolor{ansi-green}{ )} )
\end{Verbatim}
\end{tcolorbox}
        
    \begin{tcolorbox}[breakable, size=fbox, boxrule=1pt, pad at break*=1mm,colback=cellbackground, colframe=cellborder]
\prompt{In}{incolor}{34}{\boxspacing}
\begin{Verbatim}[commandchars=\\\{\}]
\PY{n+nv}{Co} \PY{o}{:=} \PY{n+nv}{CochainComplexCategory}\PY{p}{(} \PY{n+nv}{oplus} \PY{p}{)}\PY{o}{;}
\end{Verbatim}
\end{tcolorbox}

            \begin{tcolorbox}[breakable, size=fbox, boxrule=.5pt, pad at break*=1mm, opacityfill=0]
\prompt{Out}{outcolor}{34}{\boxspacing}
\begin{Verbatim}[commandchars=\\\{\}]
\textcolor{ansi-yellow}{Cochain complexes( }Additive closure( \textcolor{ansi-green}{Algebroid( }kQ/I\textcolor{ansi-green}{
)} )\textcolor{ansi-yellow}{ )}
\end{Verbatim}
\end{tcolorbox}
        
    \begin{tcolorbox}[breakable, size=fbox, boxrule=1pt, pad at break*=1mm,colback=cellbackground, colframe=cellborder]
\prompt{In}{incolor}{35}{\boxspacing}
\begin{Verbatim}[commandchars=\\\{\}]
\PY{n+nv}{Ho} \PY{o}{:=} \PY{n+nv}{HomotopyCategory}\PY{p}{(} \PY{n+nv}{oplus}\PY{o}{,} \PY{n+no}{true} \PY{p}{)}\PY{o}{;}
\end{Verbatim}
\end{tcolorbox}

            \begin{tcolorbox}[breakable, size=fbox, boxrule=.5pt, pad at break*=1mm, opacityfill=0]
\prompt{Out}{outcolor}{35}{\boxspacing}
\begin{Verbatim}[commandchars=\\\{\}]
\textcolor{ansi-yellow}{Homotopy\^{}• category( }Additive closure( \textcolor{ansi-green}{Algebroid( }kQ/I\textcolor{ansi-green}{
)} )\textcolor{ansi-yellow}{ )}
\end{Verbatim}
\end{tcolorbox}
        
    We want now to interpret the objects and generating morphisms of the
algebroid as cells concentrated at cohomological degree 0 in the
homotopy category. This can be done by defining them first in the
additive closure, then category of cochain complexes and then in the
homotopy category defined using cochain complexes.

    \begin{tcolorbox}[breakable, size=fbox, boxrule=1pt, pad at break*=1mm,colback=cellbackground, colframe=cellborder]
\prompt{In}{incolor}{36}{\boxspacing}
\begin{Verbatim}[commandchars=\\\{\}]
\PY{n+nv}{A} \PY{o}{:=} \PY{n+nv}{oid}\PY{o}{.}\PY{p}{(} \PY{l+s}{\PYZdq{}A\PYZdq{}} \PY{p}{)} \PY{o}{/} \PY{n+nv}{oplus} \PY{o}{/} \PY{n+nv}{Co} \PY{o}{/} \PY{n+nv}{Ho}\PY{o}{;}
\end{Verbatim}
\end{tcolorbox}

            \begin{tcolorbox}[breakable, size=fbox, boxrule=.5pt, pad at break*=1mm, opacityfill=0]
\prompt{Out}{outcolor}{36}{\boxspacing}
\begin{Verbatim}[commandchars=\\\{\}]
<An object in \textcolor{ansi-yellow}{Homotopy\^{}• category( }Additive closure( \textcolor{ansi-green}{Algebroid(
}kQ/I\textcolor{ansi-green}{ )} )\textcolor{ansi-yellow}{ )}>
\end{Verbatim}
\end{tcolorbox}
        
    \begin{tcolorbox}[breakable, size=fbox, boxrule=1pt, pad at break*=1mm,colback=cellbackground, colframe=cellborder]
\prompt{In}{incolor}{37}{\boxspacing}
\begin{Verbatim}[commandchars=\\\{\}]
\PY{n+nv}{Display}\PY{p}{(} \PY{n+nv}{A} \PY{p}{)}\PY{o}{;}
\end{Verbatim}
\end{tcolorbox}

    \begin{Verbatim}[commandchars=\\\{\}]
== \textcolor{ansi-yellow}{0} =======================
A formal direct sum consisting of 1 objects.
<(A)>

============================


An object in \textcolor{ansi-yellow}{Homotopy\^{}• category( }Additive closure( \textcolor{ansi-green}{Algebroid(
}kQ/I\textcolor{ansi-green}{ )} )\textcolor{ansi-yellow}{ )} given by the above data
    \end{Verbatim}

    \begin{tcolorbox}[breakable, size=fbox, boxrule=1pt, pad at break*=1mm,colback=cellbackground, colframe=cellborder]
\prompt{In}{incolor}{40}{\boxspacing}
\begin{Verbatim}[commandchars=\\\{\}]
\PY{n+nv}{B} \PY{o}{:=} \PY{n+nv}{oid}\PY{o}{.}\PY{p}{(} \PY{l+s}{\PYZdq{}B\PYZdq{}} \PY{p}{)} \PY{o}{/} \PY{n+nv}{oplus} \PY{o}{/} \PY{n+nv}{Co} \PY{o}{/} \PY{n+nv}{Ho}\PY{o}{;}\PY{o}{;}
\PY{n+nv}{C} \PY{o}{:=} \PY{n+nv}{oid}\PY{o}{.}\PY{p}{(} \PY{l+s}{\PYZdq{}C\PYZdq{}} \PY{p}{)} \PY{o}{/} \PY{n+nv}{oplus} \PY{o}{/} \PY{n+nv}{Co} \PY{o}{/} \PY{n+nv}{Ho}\PY{o}{;}\PY{o}{;}
\PY{n+nv}{D} \PY{o}{:=} \PY{n+nv}{oid}\PY{o}{.}\PY{p}{(} \PY{l+s}{\PYZdq{}D\PYZdq{}} \PY{p}{)} \PY{o}{/} \PY{n+nv}{oplus} \PY{o}{/} \PY{n+nv}{Co} \PY{o}{/} \PY{n+nv}{Ho}\PY{o}{;}\PY{o}{;}
\end{Verbatim}
\end{tcolorbox}

    \begin{tcolorbox}[breakable, size=fbox, boxrule=1pt, pad at break*=1mm,colback=cellbackground, colframe=cellborder]
\prompt{In}{incolor}{41}{\boxspacing}
\begin{Verbatim}[commandchars=\\\{\}]
\PY{n+nv}{r} \PY{o}{:=} \PY{n+nv}{oid}\PY{o}{.}\PY{p}{(} \PY{l+s}{\PYZdq{}r\PYZdq{}} \PY{p}{)} \PY{o}{/} \PY{n+nv}{oplus} \PY{o}{/} \PY{n+nv}{Co} \PY{o}{/} \PY{n+nv}{Ho}\PY{o}{;}
\end{Verbatim}
\end{tcolorbox}

            \begin{tcolorbox}[breakable, size=fbox, boxrule=.5pt, pad at break*=1mm, opacityfill=0]
\prompt{Out}{outcolor}{41}{\boxspacing}
\begin{Verbatim}[commandchars=\\\{\}]
<A morphism in \textcolor{ansi-yellow}{Homotopy\^{}• category( }Additive closure( \textcolor{ansi-green}{Algebroid(
}kQ/I\textcolor{ansi-green}{ )} )\textcolor{ansi-yellow}{ )}>
\end{Verbatim}
\end{tcolorbox}
        
    \begin{tcolorbox}[breakable, size=fbox, boxrule=1pt, pad at break*=1mm,colback=cellbackground, colframe=cellborder]
\prompt{In}{incolor}{42}{\boxspacing}
\begin{Verbatim}[commandchars=\\\{\}]
\PY{n+nv}{Display}\PY{p}{(} \PY{n+nv}{r} \PY{p}{)}\PY{o}{;}
\end{Verbatim}
\end{tcolorbox}

    \begin{Verbatim}[commandchars=\\\{\}]

== \textcolor{ansi-green}{0} =======================
A 1 x 1 matrix with entries in \textcolor{ansi-green}{Algebroid( }kQ/I\textcolor{ansi-green}{ )}

[1,1]: (A)-[\{ 1*(r) \}]->(C)

A morphism in \textcolor{ansi-yellow}{Homotopy\^{}• category( }Additive closure( \textcolor{ansi-green}{Algebroid(
}kQ/I\textcolor{ansi-green}{ )} )\textcolor{ansi-yellow}{ )} given by the above data
    \end{Verbatim}

    \begin{tcolorbox}[breakable, size=fbox, boxrule=1pt, pad at break*=1mm,colback=cellbackground, colframe=cellborder]
\prompt{In}{incolor}{49}{\boxspacing}
\begin{Verbatim}[commandchars=\\\{\}]
\PY{n+nv}{s} \PY{o}{:=} \PY{n+nv}{oid}\PY{o}{.}\PY{p}{(} \PY{l+s}{\PYZdq{}s\PYZdq{}} \PY{p}{)} \PY{o}{/} \PY{n+nv}{oplus} \PY{o}{/} \PY{n+nv}{Co} \PY{o}{/} \PY{n+nv}{Ho}\PY{o}{;}\PY{o}{;}
\PY{n+nv}{u} \PY{o}{:=} \PY{n+nv}{oid}\PY{o}{.}\PY{p}{(} \PY{l+s}{\PYZdq{}u\PYZdq{}} \PY{p}{)} \PY{o}{/} \PY{n+nv}{oplus} \PY{o}{/} \PY{n+nv}{Co} \PY{o}{/} \PY{n+nv}{Ho}\PY{o}{;}\PY{o}{;}
\PY{n+nv}{v} \PY{o}{:=} \PY{n+nv}{oid}\PY{o}{.}\PY{p}{(} \PY{l+s}{\PYZdq{}v\PYZdq{}} \PY{p}{)} \PY{o}{/} \PY{n+nv}{oplus} \PY{o}{/} \PY{n+nv}{Co} \PY{o}{/} \PY{n+nv}{Ho}\PY{o}{;}\PY{o}{;}
\PY{n+nv}{f\PYZus{}0} \PY{o}{:=} \PY{n+nv}{oid}\PY{o}{.}\PY{p}{(} \PY{l+s}{\PYZdq{}f\PYZus{}0\PYZdq{}} \PY{p}{)} \PY{o}{/} \PY{n+nv}{oplus} \PY{o}{/} \PY{n+nv}{Co} \PY{o}{/} \PY{n+nv}{Ho}\PY{o}{;}\PY{o}{;}
\PY{n+nv}{f\PYZus{}1} \PY{o}{:=} \PY{n+nv}{oid}\PY{o}{.}\PY{p}{(} \PY{l+s}{\PYZdq{}f\PYZus{}1\PYZdq{}} \PY{p}{)} \PY{o}{/} \PY{n+nv}{oplus} \PY{o}{/} \PY{n+nv}{Co} \PY{o}{/} \PY{n+nv}{Ho}\PY{o}{;}\PY{o}{;}
\PY{n+nv}{g\PYZus{}0} \PY{o}{:=} \PY{n+nv}{oid}\PY{o}{.}\PY{p}{(} \PY{l+s}{\PYZdq{}g\PYZus{}0\PYZdq{}} \PY{p}{)} \PY{o}{/} \PY{n+nv}{oplus} \PY{o}{/} \PY{n+nv}{Co} \PY{o}{/} \PY{n+nv}{Ho}\PY{o}{;}\PY{o}{;}
\PY{n+nv}{g\PYZus{}1} \PY{o}{:=} \PY{n+nv}{oid}\PY{o}{.}\PY{p}{(} \PY{l+s}{\PYZdq{}g\PYZus{}1\PYZdq{}} \PY{p}{)} \PY{o}{/} \PY{n+nv}{oplus} \PY{o}{/} \PY{n+nv}{Co} \PY{o}{/} \PY{n+nv}{Ho}\PY{o}{;}\PY{o}{;}
\end{Verbatim}
\end{tcolorbox}

    \begin{tcolorbox}[breakable, size=fbox, boxrule=1pt, pad at break*=1mm,colback=cellbackground, colframe=cellborder]
\prompt{In}{incolor}{50}{\boxspacing}
\begin{Verbatim}[commandchars=\\\{\}]
\PY{n+nv}{IsTriangulatedCategory}\PY{p}{(} \PY{n+nv}{Ho} \PY{p}{)}\PY{o}{;}
\end{Verbatim}
\end{tcolorbox}

            \begin{tcolorbox}[breakable, size=fbox, boxrule=.5pt, pad at break*=1mm, opacityfill=0]
\prompt{Out}{outcolor}{50}{\boxspacing}
\begin{Verbatim}[commandchars=\\\{\}]
true
\end{Verbatim}
\end{tcolorbox}
        
    \begin{tcolorbox}[breakable, size=fbox, boxrule=1pt, pad at break*=1mm,colback=cellbackground, colframe=cellborder]
\prompt{In}{incolor}{52}{\boxspacing}
\begin{Verbatim}[commandchars=\\\{\}]
\PY{n+nv}{st\PYZus{}r} \PY{o}{:=} \PY{n+nv}{StandardExactTriangle}\PY{p}{(} \PY{n+nv}{r} \PY{p}{)}\PY{o}{;}
\PY{n+nv}{st\PYZus{}s} \PY{o}{:=} \PY{n+nv}{StandardExactTriangle}\PY{p}{(} \PY{n+nv}{s} \PY{p}{)}\PY{o}{;}
\end{Verbatim}
\end{tcolorbox}

            \begin{tcolorbox}[breakable, size=fbox, boxrule=.5pt, pad at break*=1mm, opacityfill=0]
\prompt{Out}{outcolor}{52}{\boxspacing}
\begin{Verbatim}[commandchars=\\\{\}]
<An object in \textcolor{ansi-blue}{Category of exact triangles( }\textcolor{ansi-yellow}{Homotopy\^{}• category(
}Additive closure( \textcolor{ansi-green}{Algebroid( }kQ/I\textcolor{ansi-green}{ )} )\textcolor{ansi-yellow}{ )}\textcolor{ansi-blue}{
)}>
\end{Verbatim}
\end{tcolorbox}
        
            \begin{tcolorbox}[breakable, size=fbox, boxrule=.5pt, pad at break*=1mm, opacityfill=0]
\prompt{Out}{outcolor}{52}{\boxspacing}
\begin{Verbatim}[commandchars=\\\{\}]
<An object in \textcolor{ansi-blue}{Category of exact triangles( }\textcolor{ansi-yellow}{Homotopy\^{}• category(
}Additive closure( \textcolor{ansi-green}{Algebroid( }kQ/I\textcolor{ansi-green}{ )} )\textcolor{ansi-yellow}{ )}\textcolor{ansi-blue}{
)}>
\end{Verbatim}
\end{tcolorbox}
        
    \begin{tcolorbox}[breakable, size=fbox, boxrule=1pt, pad at break*=1mm,colback=cellbackground, colframe=cellborder]
\prompt{In}{incolor}{53}{\boxspacing}
\begin{Verbatim}[commandchars=\\\{\}]
\PY{n+nv}{Display}\PY{p}{(} \PY{n+nv}{st\PYZus{}r} \PY{p}{)}\PY{o}{;}
\end{Verbatim}
\end{tcolorbox}

    \begin{Verbatim}[commandchars=\\\{\}]
       T \^{} 0          T \^{} 1          T \^{} 2
T[ 0 ] ------> T[ 1 ] ------> T[ 2 ] ------> Σ( T[ 0 ] )

\setlength{\fboxsep}{0pt}\colorbox{ansi-cyan}{T[ 0 ]:\strut}

== \textcolor{ansi-green}{0} =======================
A formal direct sum consisting of 1 objects.
<(A)>

============================


An object in \textcolor{ansi-yellow}{Homotopy\^{}• category( }Additive closure( \textcolor{ansi-green}{Algebroid(
}kQ/I\textcolor{ansi-green}{ )} )\textcolor{ansi-yellow}{ )} given by the above data

\setlength{\fboxsep}{0pt}\colorbox{ansi-cyan}{T \^{} 0:\strut}


== \textcolor{ansi-green}{0} =======================
A 1 x 1 matrix with entries in \textcolor{ansi-green}{Algebroid( }kQ/I\textcolor{ansi-green}{ )}

[1,1]: (A)-[\{ 1*(r) \}]->(C)

A morphism in \textcolor{ansi-yellow}{Homotopy\^{}• category( }Additive closure( \textcolor{ansi-green}{Algebroid(
}kQ/I\textcolor{ansi-green}{ )} )\textcolor{ansi-yellow}{ )} given by the above data

\setlength{\fboxsep}{0pt}\colorbox{ansi-cyan}{T[ 1 ]:\strut}

== \textcolor{ansi-blue}{0} =======================
A formal direct sum consisting of 1 objects.
<(C)>

============================


An object in \textcolor{ansi-yellow}{Homotopy\^{}• category( }Additive closure( \textcolor{ansi-green}{Algebroid(
}kQ/I\textcolor{ansi-green}{ )} )\textcolor{ansi-yellow}{ )} given by the above data

\setlength{\fboxsep}{0pt}\colorbox{ansi-cyan}{T \^{} 1:\strut}


== \textcolor{ansi-blue}{0} =======================
A 1 x 1 matrix with entries in \textcolor{ansi-green}{Algebroid( }kQ/I\textcolor{ansi-green}{ )}

[1,1]: (C)-[\{ 1*(C) \}]->(C)

A morphism in \textcolor{ansi-yellow}{Homotopy\^{}• category( }Additive closure( \textcolor{ansi-green}{Algebroid(
}kQ/I\textcolor{ansi-green}{ )} )\textcolor{ansi-yellow}{ )} given by the above data

\setlength{\fboxsep}{0pt}\colorbox{ansi-cyan}{T[ 2 ]:\strut}

== \textcolor{ansi-yellow}{-1} =======================
A formal direct sum consisting of 1 objects.
<(A)>

=============================

  \textcolor{ansi-yellow}{ |}
A 1 x 1 matrix with entries in \textcolor{ansi-green}{Algebroid( }kQ/I\textcolor{ansi-green}{ )}

[1,1]: (A)-[\{ 1*(r) \}]->(C)
  \textcolor{ansi-yellow}{ |}
  \textcolor{ansi-yellow}{ V}

== \textcolor{ansi-yellow}{0} =======================
A formal direct sum consisting of 1 objects.
<(C)>

============================


An object in \textcolor{ansi-yellow}{Homotopy\^{}• category( }Additive closure( \textcolor{ansi-green}{Algebroid(
}kQ/I\textcolor{ansi-green}{ )} )\textcolor{ansi-yellow}{ )} given by the above data

\setlength{\fboxsep}{0pt}\colorbox{ansi-cyan}{T \^{} 2:\strut}


== \textcolor{ansi-yellow}{-1} =======================
A 1 x 1 matrix with entries in \textcolor{ansi-green}{Algebroid( }kQ/I\textcolor{ansi-green}{ )}

[1,1]: (A)-[\{ 1*(A) \}]->(A)
== \textcolor{ansi-yellow}{0} =======================
A 1 x 0 matrix with entries in \textcolor{ansi-green}{Algebroid( }kQ/I\textcolor{ansi-green}{ )}


A morphism in \textcolor{ansi-yellow}{Homotopy\^{}• category( }Additive closure( \textcolor{ansi-green}{Algebroid(
}kQ/I\textcolor{ansi-green}{ )} )\textcolor{ansi-yellow}{ )} given by the above data

\setlength{\fboxsep}{0pt}\colorbox{ansi-cyan}{Σ( T[ 0 ] ):\strut}

== \textcolor{ansi-yellow}{-1} =======================
A formal direct sum consisting of 1 objects.
<(A)>

=============================

  \textcolor{ansi-yellow}{ |}
A 1 x 0 matrix with entries in \textcolor{ansi-green}{Algebroid( }kQ/I\textcolor{ansi-green}{ )}

  \textcolor{ansi-yellow}{ |}
  \textcolor{ansi-yellow}{ V}

== \textcolor{ansi-yellow}{0} =======================
A formal direct sum consisting of 0 objects.

============================


An object in \textcolor{ansi-yellow}{Homotopy\^{}• category( }Additive closure( \textcolor{ansi-green}{Algebroid(
}kQ/I\textcolor{ansi-green}{ )} )\textcolor{ansi-yellow}{ )} given by the above data
    \end{Verbatim}

    \begin{tcolorbox}[breakable, size=fbox, boxrule=1pt, pad at break*=1mm,colback=cellbackground, colframe=cellborder]
\prompt{In}{incolor}{54}{\boxspacing}
\begin{Verbatim}[commandchars=\\\{\}]
\PY{n+nv}{m\PYZus{}b\PYZus{}st\PYZus{}cones} \PY{o}{:=} \PY{n+nv}{MorphismBetweenStandardConeObjects}\PY{p}{(} \PY{n+nv}{r}\PY{o}{,} \PY{n+nv}{f\PYZus{}0}\PY{o}{,} \PY{n+nv}{f\PYZus{}1}\PY{o}{,} \PY{n+nv}{s} \PY{p}{)}\PY{o}{;}
\end{Verbatim}
\end{tcolorbox}

            \begin{tcolorbox}[breakable, size=fbox, boxrule=.5pt, pad at break*=1mm, opacityfill=0]
\prompt{Out}{outcolor}{54}{\boxspacing}
\begin{Verbatim}[commandchars=\\\{\}]
<A morphism in \textcolor{ansi-yellow}{Homotopy\^{}• category( }Additive closure( \textcolor{ansi-green}{Algebroid(
}kQ/I\textcolor{ansi-green}{ )} )\textcolor{ansi-yellow}{ )}>
\end{Verbatim}
\end{tcolorbox}
        
    \begin{tcolorbox}[breakable, size=fbox, boxrule=1pt, pad at break*=1mm,colback=cellbackground, colframe=cellborder]
\prompt{In}{incolor}{55}{\boxspacing}
\begin{Verbatim}[commandchars=\\\{\}]
\PY{n+nv}{Display}\PY{p}{(} \PY{n+nv}{m\PYZus{}b\PYZus{}st\PYZus{}cones} \PY{p}{)}\PY{o}{;}
\end{Verbatim}
\end{tcolorbox}

    \begin{Verbatim}[commandchars=\\\{\}]

== \textcolor{ansi-green}{-1} =======================
A 1 x 1 matrix with entries in \textcolor{ansi-green}{Algebroid( }kQ/I\textcolor{ansi-green}{ )}

[1,1]: (A)-[\{ 1*(f\_0) \}]->(B)
== \textcolor{ansi-green}{0} =======================
A 1 x 1 matrix with entries in \textcolor{ansi-green}{Algebroid( }kQ/I\textcolor{ansi-green}{ )}

[1,1]: (C)-[\{ 1*(f\_1) \}]->(D)

A morphism in \textcolor{ansi-yellow}{Homotopy\^{}• category( }Additive closure( \textcolor{ansi-green}{Algebroid(
}kQ/I\textcolor{ansi-green}{ )} )\textcolor{ansi-yellow}{ )} given by the above data
    \end{Verbatim}

    \begin{tcolorbox}[breakable, size=fbox, boxrule=1pt, pad at break*=1mm,colback=cellbackground, colframe=cellborder]
\prompt{In}{incolor}{56}{\boxspacing}
\begin{Verbatim}[commandchars=\\\{\}]
\PY{n+nv}{f} \PY{o}{:=} \PY{n+nv}{MorphismOfExactTriangles}\PY{p}{(} \PY{n+nv}{st\PYZus{}r}\PY{o}{,} \PY{n+nv}{f\PYZus{}0}\PY{o}{,} \PY{n+nv}{f\PYZus{}1}\PY{o}{,} \PY{n+nv}{st\PYZus{}s} \PY{p}{)}\PY{o}{;}
\end{Verbatim}
\end{tcolorbox}

            \begin{tcolorbox}[breakable, size=fbox, boxrule=.5pt, pad at break*=1mm, opacityfill=0]
\prompt{Out}{outcolor}{56}{\boxspacing}
\begin{Verbatim}[commandchars=\\\{\}]
<A morphism in \textcolor{ansi-blue}{Category of exact triangles( }\textcolor{ansi-yellow}{Homotopy\^{}• category(
}Additive closure( \textcolor{ansi-green}{Algebroid( }kQ/I\textcolor{ansi-green}{ )} )\textcolor{ansi-yellow}{ )}\textcolor{ansi-blue}{
)}>
\end{Verbatim}
\end{tcolorbox}
        
    \begin{tcolorbox}[breakable, size=fbox, boxrule=1pt, pad at break*=1mm,colback=cellbackground, colframe=cellborder]
\prompt{In}{incolor}{57}{\boxspacing}
\begin{Verbatim}[commandchars=\\\{\}]
\PY{n+nv}{Display}\PY{p}{(} \PY{n+nv}{f} \PY{p}{)}\PY{o}{;}
\end{Verbatim}
\end{tcolorbox}

    \begin{Verbatim}[commandchars=\\\{\}]
A morphism of exact triangles

T[0] ------> T[1] ------> T[2] ------> Σ( T[0] )
 |            |            |              |
 | mu[0]      | mu[1]      | mu[2]        | Σ( mu[0] )
 V            V            V              V
Q[0] ------> Q[1] ------> Q[2] ------> Σ( Q[0] )

\setlength{\fboxsep}{0pt}\colorbox{ansi-cyan}{mu[ 0 ]:\strut}


== \textcolor{ansi-yellow}{0} =======================
A 1 x 1 matrix with entries in \textcolor{ansi-green}{Algebroid( }kQ/I\textcolor{ansi-green}{ )}

[1,1]: (A)-[\{ 1*(f\_0) \}]->(B)

A morphism in \textcolor{ansi-yellow}{Homotopy\^{}• category( }Additive closure( \textcolor{ansi-green}{Algebroid(
}kQ/I\textcolor{ansi-green}{ )} )\textcolor{ansi-yellow}{ )} given by the above data

\setlength{\fboxsep}{0pt}\colorbox{ansi-cyan}{mu[ 1 ]:\strut}


== \textcolor{ansi-magenta}{0} =======================
A 1 x 1 matrix with entries in \textcolor{ansi-green}{Algebroid( }kQ/I\textcolor{ansi-green}{ )}

[1,1]: (C)-[\{ 1*(f\_1) \}]->(D)

A morphism in \textcolor{ansi-yellow}{Homotopy\^{}• category( }Additive closure( \textcolor{ansi-green}{Algebroid(
}kQ/I\textcolor{ansi-green}{ )} )\textcolor{ansi-yellow}{ )} given by the above data

\setlength{\fboxsep}{0pt}\colorbox{ansi-cyan}{mu[ 2 ]:\strut}


== \textcolor{ansi-magenta}{-1} =======================
A 1 x 1 matrix with entries in \textcolor{ansi-green}{Algebroid( }kQ/I\textcolor{ansi-green}{ )}

[1,1]: (A)-[\{ 1*(f\_0) \}]->(B)
== \textcolor{ansi-magenta}{0} =======================
A 1 x 1 matrix with entries in \textcolor{ansi-green}{Algebroid( }kQ/I\textcolor{ansi-green}{ )}

[1,1]: (C)-[\{ 1*(f\_1) \}]->(D)

A morphism in \textcolor{ansi-yellow}{Homotopy\^{}• category( }Additive closure( \textcolor{ansi-green}{Algebroid(
}kQ/I\textcolor{ansi-green}{ )} )\textcolor{ansi-yellow}{ )} given by the above data

\setlength{\fboxsep}{0pt}\colorbox{ansi-cyan}{mu[ 3 ]:\strut}


== \textcolor{ansi-magenta}{-1} =======================
A 1 x 1 matrix with entries in \textcolor{ansi-green}{Algebroid( }kQ/I\textcolor{ansi-green}{ )}

[1,1]: (A)-[\{ 1*(f\_0) \}]->(B)
== \textcolor{ansi-magenta}{0} =======================
A 0 x 0 matrix with entries in \textcolor{ansi-green}{Algebroid( }kQ/I\textcolor{ansi-green}{ )}


A morphism in \textcolor{ansi-yellow}{Homotopy\^{}• category( }Additive closure( \textcolor{ansi-green}{Algebroid(
}kQ/I\textcolor{ansi-green}{ )} )\textcolor{ansi-yellow}{ )} given by the above data
    \end{Verbatim}

    \begin{tcolorbox}[breakable, size=fbox, boxrule=1pt, pad at break*=1mm,colback=cellbackground, colframe=cellborder]
\prompt{In}{incolor}{58}{\boxspacing}
\begin{Verbatim}[commandchars=\\\{\}]
\PY{n+nv}{g01} \PY{o}{:=} \PY{n+nv}{HomotopyCategoryMorphism}\PY{p}{(} 
            \PY{n+nv}{st\PYZus{}r}\PY{p}{[} \PY{n+nv}{2} \PY{p}{]}\PY{o}{,}
            \PY{n+nv}{st\PYZus{}s}\PY{p}{[} \PY{n+nv}{2} \PY{p}{]}\PY{o}{,}
            \PY{p}{[} \PY{n+nv}{oid}\PY{o}{.}\PY{p}{(} \PY{l+s}{\PYZdq{}g\PYZus{}0\PYZdq{}} \PY{p}{)} \PY{o}{/} \PY{n+nv}{oplus}\PY{o}{,} \PY{n+nv}{oid}\PY{o}{.}\PY{p}{(} \PY{l+s}{\PYZdq{}g\PYZus{}1\PYZdq{}} \PY{p}{)} \PY{o}{/} \PY{n+nv}{oplus} \PY{p}{]}\PY{o}{,}
            \PY{o}{\PYZhy{}}\PY{n+nv}{1}
        \PY{p}{)}\PY{o}{;}
\end{Verbatim}
\end{tcolorbox}

            \begin{tcolorbox}[breakable, size=fbox, boxrule=.5pt, pad at break*=1mm, opacityfill=0]
\prompt{Out}{outcolor}{58}{\boxspacing}
\begin{Verbatim}[commandchars=\\\{\}]
<A morphism in \textcolor{ansi-yellow}{Homotopy\^{}• category( }Additive closure( \textcolor{ansi-green}{Algebroid(
}kQ/I\textcolor{ansi-green}{ )} )\textcolor{ansi-yellow}{ )}>
\end{Verbatim}
\end{tcolorbox}
        
    \begin{tcolorbox}[breakable, size=fbox, boxrule=1pt, pad at break*=1mm,colback=cellbackground, colframe=cellborder]
\prompt{In}{incolor}{59}{\boxspacing}
\begin{Verbatim}[commandchars=\\\{\}]
\PY{n+nv}{IsWellDefined}\PY{p}{(} \PY{n+nv}{g01} \PY{p}{)}\PY{o}{;}
\end{Verbatim}
\end{tcolorbox}

            \begin{tcolorbox}[breakable, size=fbox, boxrule=.5pt, pad at break*=1mm, opacityfill=0]
\prompt{Out}{outcolor}{59}{\boxspacing}
\begin{Verbatim}[commandchars=\\\{\}]
true
\end{Verbatim}
\end{tcolorbox}
        
    \begin{tcolorbox}[breakable, size=fbox, boxrule=1pt, pad at break*=1mm,colback=cellbackground, colframe=cellborder]
\prompt{In}{incolor}{60}{\boxspacing}
\begin{Verbatim}[commandchars=\\\{\}]
\PY{n+nv}{g} \PY{o}{:=} \PY{n+nv}{MorphismOfExactTriangles}\PY{p}{(}
            \PY{n+nv}{st\PYZus{}r}\PY{o}{,}
            \PY{n+nv}{f}\PY{p}{[} \PY{n+nv}{0} \PY{p}{]}\PY{o}{,}
            \PY{n+nv}{f}\PY{p}{[} \PY{n+nv}{1} \PY{p}{]}\PY{o}{,}
            \PY{n+nv}{g01}\PY{o}{,}
            \PY{n+nv}{st\PYZus{}s}
        \PY{p}{)}\PY{o}{;}
\end{Verbatim}
\end{tcolorbox}

            \begin{tcolorbox}[breakable, size=fbox, boxrule=.5pt, pad at break*=1mm, opacityfill=0]
\prompt{Out}{outcolor}{60}{\boxspacing}
\begin{Verbatim}[commandchars=\\\{\}]
<A morphism in \textcolor{ansi-blue}{Category of exact triangles( }\textcolor{ansi-yellow}{Homotopy\^{}• category(
}Additive closure( \textcolor{ansi-green}{Algebroid( }kQ/I\textcolor{ansi-green}{ )} )\textcolor{ansi-yellow}{ )}\textcolor{ansi-blue}{
)}>
\end{Verbatim}
\end{tcolorbox}
        
    \begin{tcolorbox}[breakable, size=fbox, boxrule=1pt, pad at break*=1mm,colback=cellbackground, colframe=cellborder]
\prompt{In}{incolor}{61}{\boxspacing}
\begin{Verbatim}[commandchars=\\\{\}]
\PY{n+nv}{IsWellDefined}\PY{p}{(} \PY{n+nv}{g} \PY{p}{)}\PY{o}{;}
\end{Verbatim}
\end{tcolorbox}

            \begin{tcolorbox}[breakable, size=fbox, boxrule=.5pt, pad at break*=1mm, opacityfill=0]
\prompt{Out}{outcolor}{61}{\boxspacing}
\begin{Verbatim}[commandchars=\\\{\}]
true
\end{Verbatim}
\end{tcolorbox}
        

    % Add a bibliography block to the postdoc
    
    
    
\end{document}
